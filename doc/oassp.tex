\section{OASSP: 2-D Waveguide Reverberation Realizations}

    The OASES-OASSP module is the time domain equivalent of the OASS
reverberation model. However, in contrast to OASS which directly
generates expectation values for the scattered field correlation,
OASSP generates timeseries using realizations of the environmental
perturbations. Earlier versions of OASSP allowed for rough interfaces
with Gaussian or power law power spectra. Version 2.2 and later in
addition allows for generating field realizations for scattering from
volume inhomogeneities in fluid layers with or without sound speed
gradients, based on the theoretical developments of LePage and Schmidt
\cite{Lepage_99}. OASSP is used in conjunction with OASP, which is
used for computing the mean, coherent field, which is driving the
scattering. The data files for OASSP are very similar to those of
OASP, and OASSP generates a transfer function file in the same format,
which is post-processed and plotted using PP.  As is the case for all
the other modules, both stresses and particle velocities can be
computed.

\subsection{Transfer Functions}

Similar to OASP, in addition to generating timeseries,
OASSP may be used for generating the complex CW reverberant field over a
rectangular grid in range and depth. Here again it should be noted
that when OASSP is used with option 'O' or with automatic
sampling enabled, the transfer functions are computed for complex
frequencies. Complex frequency corresponds to applying a {\em
time-domain damping} which cannot be directly compensated for in the
transfer functions. However, real frequencies can be forced in
automatic sampling mode by using option 'J' (Version 2.1 and
later). In any case the frequency sampling must be consistent for the
computation of the mean and scattered fields.

\subsection{Input Files for OASSP}

The input files for OASSP are virtually identical to the ones used for
computing the mean field using OASP. A few of the options have
different significance ({\bf s}), and the option {\bf p } is unique to
OASSP. In addition, OASSP is unique in terms of allowing for
computation of the reverberation from sound speed inhomogeneities in
fluid layers. The spatial statistics of the volume inhomogeneities is
controlled by the interface roughness parameters for such layers.
The input files for OASSP are structured in 8 blocks,
as outlined in Table\,\ref{tab:oasspI}. 
\begin{table}
\begin{center}
\small
\begin{tabular}{|l|l|c|c|}
\hline \hline
Input parameter & Description & Units & Limits \\
\hline \hline
\multicolumn{4}{|l|}{\bf BLOCK I: TITLE } \\
\hline
TITLE & Title of run  & - & $\leq$ 80 ch. \\
\hline
\multicolumn{4}{|l|}{\bf BLOCK II: OPTIONS} \\
\hline
A B C $\cdots$ & Output options & - & $\leq$ 40 ch. \\
\hline
\multicolumn{4}{|l|}{\bf BLOCK III: SOURCE FREQUENCY} \\
\hline
FRC,COFF,IT,VS,VR & FRC: Center frequency of source & Hz & $>0$ \\
 	& COFF: Integration contour offset & dB/$\Lambda$ & COFF$\geq 0$ \\
        & IT: Source pulse type (only for option d) & & \\
	& VS,VR: Sou./Rec. velocity  (only for option d) & m/s & \\
\hline
\multicolumn{4}{|l|}{\bf BLOCK IV: ENVIRONMENT} \\
\hline
NL 	& Number of layers, incl. halfspaces	& - & NL$\geq 2$  \\
D,CC,CS,AC,AS,RO,RG,CL & D: Depth of interface. & m & - \\
.	& CC: Compressional speed & m/s & CC$\geq 0$ \\
.	& CS: Shear speed & m/s & - \\
.	& AC: Compressional attenuation & dB/$\Lambda$ & AC$\geq 0$ \\
.	& AS: Shear attenuation & dB/$\Lambda$ & AS$\geq 0$ \\
	& RO: Density 	& g/cm$^{3}$ & RO$\geq 0$ \\
	& RG: RMS value of interface roughness & m & - \\
	& CL: Correlation length of roughness & m & - \\
	& M:  Spectral exponent &   & > 1.5 \\
\hline
\multicolumn{4}{|l|}{\bf BLOCK V: SOURCES} \\
\hline
SD,NS,DS,AN,IA,FD,DA & SD: Source depth (mean for array) & m & - \\
	& NS: Number of sources in array & - &  NS$>0$ \\
	& DS: Vertical source spacing	& m & DS$>0$ \\
	& AN: Grazing angle of beam & deg & - \\
	& IA: Array type & - & $1 \leq $IA$\leq 5$ \\
	& FD: Focal depth of beam & m & FD$\neq$SD \\
        & DA: Dip angle. (Source type 4). & deg & - \\
\hline
\multicolumn{4}{|l|}{\bf BLOCK VI: RECEIVER DEPTHS} \\
\hline
RD1,RD2,NRD & RD1: Depth of first receiver & m & - \\
	& RD2: Depth of last receiver  & m & RD2$>$RD1 \\
	& NRD: Number of receiver depths & - & NR$>0$ \\
\hline
\multicolumn{4}{|l|}{\bf BLOCK VII: WAVENUMBER SAMPLING} \\
\hline
CMIN,CMAX & CMIN: Minimum phase velocity & m/s & CMIN$>0$ \\
	& CMAX: Maximum phase velocity & m/s & - \\
NW,IC1,IC2,IF & NW: Number of wavenumber samples & - & $>0$, -1 (auto) \\
	& IC1: First sampling point & - & IC1$\geq 1$ \\
	& IC2: Last sampling point & - & IC2$\leq$NW \\
        & IF: Freq. sample increment for kernels & & $\geq 0$ \\
\hline
\multicolumn{4}{|l|}{\bf BLOCK VIII: FREQUENCY AND RANGE  SAMPLING} \\
\hline
NT,FR1,FR2,DT,R1,DR,NR & NT: Number of time samples & & NT$=2^{M}$ \\
	& FR1: lower limit of frequency band & Hz & $\geq 0$ \\
	& FR2: upper limit of frequency band & Hz & $\geq$FR1 \\
        & DT: Time sampling increment & s & $> 0$ \\ 
        & R1: First range & km & \\
        & DR: Range increment & km & \\
	& NR: Number of ranges & & $> 0$ \\
\hline
\end{tabular}
\end{center}
\caption{Layout of OASSP input files: Computational parameters.
	\label{tab:oasspI} }
\end{table} 

\subsection{Block I: Title}

The title printed on all graphic output generated by OASSP.

\subsection{Block II: OASSP options}

The significance of the single-letter OASSP options are:
\begin{itemize}
    \item[{\bf C}] Creates an $ \omega - k$ representation of the
      field in the form of contours of integration kernels as function
      of horizontal wavenumber (slowness if option {\bf B} is
      selected) and frequency (logarithmic y-axis). All axis
      parameters are determined automatically.  
\item[{\bf G}] Rough
      interfaces are assumed to be characterized by a Goff-Jordan
      power spectrum rather than the default Gaussian.  
\item[{\bf H}]
      Horizontal (radial) particle velocity calculated.  
\item[{\bf J}] Complex wavenumber contour. The contour is shifted into the
      upper halfpane by an offset controlled by the input parameter
      COFF (Block III). NOTE: If this option is used together with
      automatic sampling, the complex frequency integration (option
      {\bf O}) is disabled, allowing for computation of complex CW
      fields or transmission losses (plotted using PP). 
\item[{\bf K}]
      Computes the bulk stress. In elastic media the bulk stress only
      has contributions from the compressional potential. In fluid
      media the bulk stress is equal to the negative of the pressure.
      Therefore for fluids this option yields the same result as
      option N or R.  
\item[{\bf L}] Linear vertical source array.
\item[{\bf N}] Normal stress $\sigma_{zz}$ ($=-p$ in fluids)
      calculated.  
\item[{\bf O}] Complex frequency integration
      contour. This new option is the frequency equivalent of the
      complex wavenumber integration ({\bf J} option in OAST). It
      moves the frequency contour away from the real axis by an amount
      reducing the time domain wrap-around by a factor 50
      \cite{jkps}. This option can yield significant computational
      savings in cases where the received signal has a long time
      duration, and only the initial part is of interest, since it
      allows for selection of a time window shorter than the actual
      signal duration. Note that only wrap around from later times is
      reduced; therefore the time window should always be selected to
      contain the beginning of the signal!  
\item[{\bf P}] Plane
      geometry. The sources will be line-sources instead of
      point-sources as used in the default cylindrical geometry.
\item[{\bf R}] Computes the radial normal stress $\sigma_{rr}$
      (or $\sigma_{xx}$ for plane geometry).  \item[{\bf S}] Computes
      the stress equivalent of the shear potential in elastic
      media. This is an angle-independent measure, proportional to the
      shear potential, with no contribution from the compressional
      potential (incontrast to shear stress on a particular plane).
      For fluids this option yields zero.  
\item[{\bf T}] The new
      option `T' allows for specification of an array tilt in the
      vertical plane containing the source and the receivers.  See
      below for specification of array tilt parameters.  
\item[{\bf U}] Decomposed seismograms.  This option generates 5 transfer
      function files to be processed by PP:

          \begin{tabular}{ll}
          File name & Contents \\ \hline
          input.trf &     Complete transfer functions \\
          input.trfdc &    Downgoing compressional waves \\
          input.trfuc  &  Upgoing compressional waves  \\
          input.trfds  &  Downgoing  shear waves alone \\
          input.trfus &    Upgoing shear waves
          \end{tabular}
\item[{\bf V}] Vertical particle velocity calculated.
\item[{\bf Z}] Plot of SVP will be generated.

\item[{\bf d}] Radial {\em Doppler shift} is accounted for by
specifying  this option, using the theory developed by Schmidt and Kuperman
\cite{sk:jasa94}.  The source pulse and the radial projections of the
source and  receiver velocities must be specified in the input file
following the specification of the centre frequency and the contour
offset (Block II). Since this option requires incorporation of the
source function in the wavenumber integral, the PP post-processor
\underline{must} be used with source pulse -1 (impulse response).

    \item[{\bf f}] Full Bessel function integration. This new option
          does not apply the asymptotic representation of the Bessel
          function in the evaluation of the inverse Hankel transforms.
          The implementation is very efficient, and the integral
          evaluation is performed just as fast as the asymptotic
          evaluations. 
          
\item[{\bf g}] Rough interfaces and volume inhomogeneities
          are assumed to be characterized by a Goff-Jordan power
          spectrum rather than the default Gaussian (Same as G).
          
\item[{\bf l}] User defined source array. This new option is
          similar to option {\bf L} in the sense that that it
          introduces a vertical source array of time delayed sources
          of identical type. However, this option allows the depth,
          amplitude and delay time to be be specified individually for
          each source in the array. The source data should be provided
          in a separate file, {\bf input.src}, in the format described
          in Section~\ref{oaspsar}.  
\item[{\bf p}] The perturbed Green's functions will be used for the
          waveguide propagation of the scattered field. This
          corresponds to applying scattering loss to the propagation
          and therefore constitutes a higher order scattering term,
          representing multiple scattering. In general {\bf p} will yield
          a lower bound for the scattered field, while the default
          yields an upper bound. Option {\bf p} has no effect for
          volume scattering, which is always computed in the Born
          approximation. 
\item[{\bf s}]
          OASSP will compute the scattered field perturbation
          alone. By default the total field will be calculated.
\item[{\bf t}] Eliminates the wavenumber integration and
          computes transfer functions for individual slowness
          components (or plane wave components). The Fourier transform
          performed in PP will then directly compute the
          slowness/intercept-time or $\tau - p$ response for each of
          the selected depths. When option {\bf t} is selected, the
          range parameters in the data file are insignificant.
\item[{\bf v}] As option {\bf l} this option allows for
          specifying a non-standard source array. However, it is more
          general in the sense that different types of sources can be
          applied in the same array, and the sources can have
          different signatures. The array geometry and the complex
          amplitudes are specified in a file {\bf input.strf} which
          should be of {\bf trf} format as described in
          Section~\ref{oaspsar}.  \item[{\bf \#}] Number $(1-5)$
          specifying the source type (explosive, forces, seismic
          moment) as described in Section~\ref{oaspsou}
\end{itemize}


\subsubsection{Block III: Source Frequency}

FRC is the source center frequency. As the source convolution is
performed in PP,  FRC is not used in OASSP, except for cases involving
moving sources (option {\bf d}), but will be written to the
transfer function file and become the default for PP.

COFF is the complex wavenumber
integration 
contour offset. To be specified in
		$dB/\lambda$, where $\lambda$ is the wavelength at the source 
		depth SD. As only the horizontal part of
		the integration contour is considered, this parameter
		should not be chosen so large, that the amplitudes
		at the ends of the integration interval become 
		significant. In lossless cases too small values
		will give sampling problems at the normal modes
		and other singularities. For intermediate values,
		the result is independent of the choice of COFF,
		but a good value to choose is one that gives 60 dB
		attenuation at the longest range considered in the FFT, i.e.
                \begin{displaymath}
		\mbox{COFF} = \frac{60 \ast \mbox{CC(SD)}}{( \mbox{FREQ} \ast \mbox{R}_{max} )}
		\end{displaymath} 
       		where the maximum FFT range is
       		\begin{displaymath}
		\mbox{R}_{max} = \frac{\mbox{NP}}{\mbox{FREQ}\ast(1/\mbox{CMIN} - 1/\mbox{CMAX})}
		\end{displaymath}
		This value is the default which is applied if COFF
		is specified to 0.0.

\noindent{\bf Doppler shift}

By specifying option {\bf d} in OASP V.1.7 and higher,
radial {\em Doppler shift} is accounted for 
using the theory developed by Schmidt and Kuperman
\cite{sk:jasa94}.  The source pulse and the radial projections of the
source and  receiver velocities must be specified in the input file
following the specification of the centre frequency and the contour
offset (Block II), i.e.

\begin{tabular}{cc}
   Standard   &    For option {\bf d} \\ \hline
FRC COFF & IT VS VR
\end{tabular}


\tt IT \rm is a number identifying the source pulse as described
in Sec.\,\ref{postproc}. \tt VS \rm and \tt VR \rm are the
projected radial velocities in m/s of the source and receiver, respectively,
both being positive in the direction from source to receiver.
Since this option requires incorporation of the
source function in the wavenumber integral, the PP post-processor
\underline{must} subsequently be used with source pulse -1 (impulse response).


\subsubsection{Block IV: Environmental Model}

OASSP supports all the environmental models allowed for SAFARI as well as the
ones described above in Section~\ref{oas_env}.  The significance of
the standard environmental parameters is as follows
\begin{itemize}	
		\item[NL:]	Number of layers, including the upper and lower
		half-spaces. These should always be included,
		even in cases where they are vacuum.

		\item[D:]	Depth in $m$ of upper boundary of layer or
		halfspace. The reference depth can be choosen
		arbitrarily, and D() is allowed to be negative.
		For layer no. 1, i.e. the upper half-space, this
     		parameter is dummy.

		\item[CC:]	Velocity of compressional waves in $m/s$.
        	If specified to 0.0, the layer or half-space is
		vacuum.

		\item[CS:]	Velocity of shear waves in m/s.
 		If specified to 0.0, the layer or half-space is fluid.
                If CS()$< 0$, it is the compressional velocity at bottom of
		layer, which is treated as fluid with $1/c(z)^{2}$ linear.

		\item[AC:]   Attenuation of compressional waves in 
		$dB/\lambda$. If the layer is fluid, and AC() is specified to
		0.0, then an imperical water attenuation is
		used (Skretting \& Leroy).

		\item[AS:]   Attenuation of shear waves in $dB/\lambda$

		\item[RO:]   Density in $g/cm^{3}$.

		\item[RG:] RMS roughness of interface in $m$. RG(1) is
		dummy. If RG$<0$ it represents the negative of the RMS
		roughness, and the associated correlation length CL and the spectral exponent 
		should follow. If RG$>0$ the correlation length is
		assumed to be infinite.  

                \item[CL:] Roughness correlation length in m. If
		RG and CL are both negative, the interface is assumed to be
		smooth, but the layer below will contain volume
		inhomogeneities with vertical and horizontal
		correlation lengths -RG and -CL, respectively. 

                \item[M:] Spectral exponent of the power spectrum as
                defined by Turgut \cite{Turgut_97}, with $1.5 <
                \mbox{M} \le 2.5 $ for realistic surfaces, with
                $\mbox{M} =1.5 $ corresponding to the highest
                roughness, and $\mbox{M}=2.5$ being a very smooth
                variation. For 2-D Goff-Jordan surfaces, the fractal
                dimension is $ \mbox{D} = 4.5 - \mbox{M} $
                Insignificant
                for Gaussian spectrum (option {\bf g} not specified)
                but a  value must
                be given.
\end{itemize}

\subsubsection{Volume Scattering Layer}
If a volume scattering layer is flagged by RG and CL both being
		negative, these parameters will be interpreted as the
		negative of the vertical and horizontal correlation
		lengths of the volume inhomogeneities. In this
		case 4 more parameters should be added on the same
		line to completely describe the spatial statistics:
\begin{verbatim}
D CC CS AC AS RO RG CL SKW M RMS GAM
\end{verbatim}
with the significance:
 \begin{itemize}
 \item[SKW] Skewness angle of the correlation ellipse in degrees. 0.0
 represents no skewness. Angle measured positive downwards.
 \item[M] Volume spectral exponent as defined by Turgut
 \cite{Turgut_97} for power-law spectra.  $ \mbox{M} > 1.5 $ is
 required for the power spectrum to be integrable, independent on the
 dimension of the problem. Insignificant
 unless option {\bf g} was specified.
 \item[RMS] RMS of relative sound speed perturbation $\Delta c / c$ for
 volume inhomogeneities.
 \item[GAM] Ratio $\gamma$ of relative density perturbation to sound speed
 perturbation, $ \Delta \rho / \rho = 2.0 \gamma \; \Delta c / c$.
 \end{itemize}



\subsubsection{Block V: Sources}

When computing the total field (option {\bf s} not specified), OASSP
supports the same sources as OASP, as described in
Sec.\,\ref{oaspsou}, including source arrays. However, the source
specification must be consistent with the one used for the associated
OASP run, since the scattered field is controlled by the latter.  If
option {\bf s} is specified in OASSP, the source type is
insignificant.


\subsubsection{Block VI: Receivers}

The default specification of the receiver depths is the same as for
SAFARI, i.e. through the parameters RD1, RD2 and NRD in Block VI, with
\begin{itemize}
\item[RD1]  Depth of uppermost receiver in meters
\item[RD2]  Depth of lowermost receiver in meters
\item[NRD]   Number of receiver depths 
\end{itemize}

The NRD receivers are placed equidistantly in the vertical.

\noindent {\bf Non-equidistant Receiver Depths}

In OASES the receiver depths can optionally be specified individually.
The parameter NRD is used as a flag for this option. Thus, if NRD $< 0$
the number of receivers is interpreted as --NRD, with the individual
depths following immidiately following Block VI. 

The PP Post-processor is compatible and will depth-stack the traces at
the correct depths.

\noindent {\bf Tilted Receiver Arrays}

Option `T' allows for specification of an array tilt in the
vertical plane containing the source and the receivers.

The tilt angle and rotation origin is specified in the receiver depth
line (Block VI):

\begin{tabular}{cc}
   Standard   &    For option T \\ \hline
RD1 RD2 NR &  ZREF  ANGLE
\end{tabular}


The vertical arrays are rotated by an angle `ANGLE' in deg relative
to the vertical. The rotation is performed with origin at depth `ZREF'.

The parameters RD1, RD2 and NR always refer to the untilted case. 
In the tilted case these parameters do therefore define the array
geometry and not the actual depths of the receivers in the tilted array.
The same is the case for the graphics output produced by the
post-processor PP. 


The source(s) is always at the origin and is therefore not rotated.
Thus, for zero-offset tilted VSP-s, the reference depth ZREF should be set 
equal to the source depth SD!



\subsubsection{Block VII: Wavenumber integration}

This block specifies the wavenumber sampling in the standard OASES
format, with the significance of the parameters being as follows:
\begin{itemize}
		\item[CMIN:]   Minimum phase velocity in m/s. Determines the
		upper limit of the truncated horizontal wavenumber space:
			\begin{displaymath}
			k_{max} = \frac{2\pi \ast \mbox{FREQ}}{\mbox{CMIN}}
			\end{displaymath}
		\item[CMAX:]	Maximum phase velocity in m/s. Determines the
		lower limit of the truncated horizontal wave-
		number space:
			\begin{displaymath}
			k_{min} = \frac{2\pi \ast \mbox{FREQ}}{\mbox{CMAX}}
			\end{displaymath}
		In plane geometry ( option P ) CMAX may be specified as 
		negative. In this case, the negative wavenumber spectrum
		will be included with $k_{min}=-k_{max}$, yielding correct 
		solution also at zero range.
		\item[NW:]	Number of sampling points in wavenumber space.
		In contrast to what is the case for OAST, 
		NW does here not have to be an integer  power of 2.				The sampling points are placed equidistantly
		in the truncated wavenumber space determined
		by CMIN and CMAX. If CMAX$<0$, i.e. the inclusion of
		the negative spectrum is enabled, then the NW sample
		points will be distributed along the positive
		wavenumber axis only, with the negative
		components obtained by symmetry. NW=-1 activates
		automatic wavenumeber sampling.

		\item[IC1:]  Number of the first sampling point where the
		calculation is to be performed. If IC1$>1$, 
		then the Hankel transform is Hanning-windowed in the
		interval 1,2$\ldots$IC1-1 before
		integration.  Insignificant for automatic sampling (NW
		= -1).

		\item[IC2:]	Number of the last sampling point where the 
		calculation is to be performed. If IC2$<$NWN,
		then the Hankel transform is Hanning windowed in the
		interval IC2+1,$\ldots$NW before integration. Insignificant for automatic sampling (NW
		= -1).

		\item[IF:] Frequency increment for plotting of
integration  kernels. A value of 0 disables the plotting.
		\end{itemize}
 
\noindent {\bf Automatic wavenumber sampling}



As foor the other OASES modules the  automatic sampling  is  activated  by 
specifying the parameter NW to $ -1$ and it automatically  activates 
the  complex frequency integration contour even though option  {\bf O} 
may  not have been specified. The parameters IC1 and IC2 have  no 
effect if the automatic sampling is selected.
 
\subsection{Execution of OASSP}

    As  for  the other modules  filenames  are  passed  to  the  code   via 
environmental parameters. In Unix systems a typical command  file 
{\bf oassp} (in  \$HOME/oases/bin) is:

\small
\begin{verbatim}
#!/bin/csh
    setenv FOR001 $1.dat       # input file 
    setenv FOR002 $1.src       # Source array input file
    setenv FOR015 $1.strf      # Source array trf file
    setenv FOR019 $1.plp       # plot parameter file
    setenv FOR020 $1.plt       # plot data file  
    setenv FOR028 $1.cdr       # contour plot parameter file 
    setenv FOR029 $1.bdr       # contour plot data file 
    setenv FOR045 $2.rhs       # mean field amplitudes at rough interface
    setenv FOR046 $2.vol       # mean field in volume scattering layer
    oassp2                      # executable
\end{verbatim}
\normalsize

    After preparing an OASP  data file with the name \tt input-rhs.dat
\rm, OASP  is 
executed by the command:

    $>$ {\bf oasp input-rhs}

To generate the associated scattered field transfer functions, OASSP
    is then executed,

    $>$ {\bf oassp input-scat input-rhs}

where \tt input-scat.dat \rm is the input file for OASSP. 

\newpage
\subsection{OASSP - Examples}

\subsubsection{Rough Seabed Reverberation}

This example computes the reverberation in a 200 Hz band around 500 Hz
from a rough stratified seabed in shallow water. The source is assumed
to be a line source (option P). The field is computed for vertical
arrays at range 0 m (monostatic) and 500 m (bistatic). The data file
\tt wg-rhs.dat \rm for computing the mean field using automatic
sampling is:

\begin{verbatim}
Line source. Mean field. Goff. auto.
P B N s g O 
500 0
4
0          0      0     0     0     0      0
0       1500      0     0     0     1      0 
128.    1470   0000.00  0.01  0.00  1.650  0.5  0
228.00  2300   0000.00  0.50  0.00  2.650  0.0000  	

100.
52. 127. 51

1200 -1200
-1 1 1 0
2048 400 600 0.0004 0 .5 2 
\end{verbatim}

\figmit{figs/wg_rhs.ps}{7.0}{Mean field envelope in shallow water
example}{fig:wg_mean}

Figure\,\ref{fig:wg_mean} shows the depth-stacked plots of the the mean
field envelopes for the monostatic receiver array, generated using PP. 

The associated input file \tt wg-scat.dat \rm for OASSP is:

\begin{verbatim}
Seabed scat. Line source. Goff-J. auto.
P B N O s g  
500 0
4
  0.       0      0     0     0     1      0       0
  0.    1500      0     0     0     1      0       0
128.    1470   0000.00  0.01  0.00  1.650  -0.5   5.0 
228.00  2300   0000.00  0.50  0.00  2.650  0.0000 0.0 	

128.
77. 127. 51

1200 -1200
2048 1 2048 0
2048 400 600 0.0004 0 0.5 2
\end{verbatim}

\figmit{figs/wg_scat.ps}{7.0}{Reverberant  field envelope in shallow water
seabed roughness scattering example}{fig:wg_scat}

The main difference is that the wavenumber sampling was chosen
manually here, and a finite roughness correlation length was
specified. The field envelopes for the monostatic vertical array are
shown in Fig.\,\ref{fig:wg_scat}.

\subsubsection{Seabed Volume Scattering}

The use of OASSP for computing realizations of scattering and
reverberation produced by random distributions of volume
inhomogeneities in the seabed is illustrated by the following
example. The environment is similar to the above, but the seasurface
is eliminated. The data file \tt vol-rhs.dat \rm for the OASP mean field is 

\begin{verbatim}
Halfsp. Line source. Mean field. Goff. auto.
P B N s g O 
500 0
3
128.    1500      0     0     0     1      0       0
128.    1470  -2270.10  0.01  0.00  1.650  1.0000  0
228.00  2300   0000.00  0.50  0.00  2.650  0.0000  0	

100.
77. 127. 51

1200 -1200
-1 1 1 0
2048 400 600 0.0004 0 .5 2
\end{verbatim}

\figmit{figs/vol_rhs.ps}{7.0}{Monostatic mean field envelope in shallow water
environment with sediment layer having a strong sound speed gradient
and volume inhomogeneities}{fig:vol_mean}

The resulting depth-stacked envelope plot is shown in
Fig.\,\ref{fig:vol_mean}.

The associated OASSP data file \tt vol-scat.dat \rm for calculating
the field scattered is

\begin{verbatim}
Volume scat. Line source. . Goff-J. auto.
P N s O g  
500 0
3
128.    1500      0     0     0     1       0    0.0
128.    1470  -2270.10  0.01  0.00  1.650  -2.0 -5.0 0 2.5 1.0 0.0
228.00  2300   0000.00  0.50  0.00  2.650  0.0000 0.0 	

128.
77. 127. 51

1200 -1200
2048 1 2048 0
2048 400 600 0.0004 0 0.5 2
\end{verbatim}

With the skewness set to 0 deg the main axes of the correlation function
are alligned with the stratification, and the horizontal correlation
length is specified to 5 m, with the vertical  correlation length
being 2 m. The fractal dimension of the power law spectrum (option
'g') is 2.5. The RMS relative sound speed perturbation is set to unity,
while the relative density is set to 0. 

\figmit{figs/vol_scat.ps}{7.0}{Monostatic backscattering envelope in
shallow water environment with sediment layer having a strong sound
speed gradient and volume inhomogeneities}{fig:vol_scat}

The resulting depth-stacked scattered field envelopes are shown in
Fig.\,\ref{fig:vol_scat}. The fundamental differences between the
scattering from seabed roughness and volume inhomogeneities is
didcussed in Ref.~\cite{Lepage_99}.

    

