\section{OASP: 2-D Wideband Transfer Functions}

    The OASES-OASP module calculates the depth-dependent  Green's 
function for a selected number of frequencies and determines  the 
transfer function at any receiver position by evaluating  the 
wavenumber  integral. The frequency integral is evaluated in the
Post-processor PP.  As is  the 
case for OASES-OAST, both stresses and particle velocities can be 
determined, and the field may be produced by either point or line 
sources.  By  arranging the sources in a vertical  phased  array, 
pulsed beam propagation can be analyzed.

\subsection{Two-Step Execution}

    Whereas its predecessor SAFARI-FIPP was always run in a
\underline{one-step} 
mode,  directly generating the time responses on the receiver 
array corresponding to the chosen source function, OASES-OASP 
is always  run in a \underline{two-step} mode in conjunction with  the  PP 
Pulse  Post-processor.  OASP   generates  the 
transfer  functions  for  the selected  environment  and  source-
receiver geometry. The interactive postprocessor is then used  to 
select time windows, source pulses, stacking format etc. 

    Since the computation of the transfer functions is by far the 
most  computationally  intensive  part  of  synthetic  seismogram 
computation,  the  2-step  mode will be most  user  friendly  and 
efficient,  except  for cases where large  batches  of  synthetic 
seismograms have to be generated. The transfer function file will have
the same name as the input file, but extension {\bf .trf}, i.e. for an
input file {\bf input.dat}, the transfer function file will be 

{\bf
input.trf}. 

\subsection{Transfer Functions}

In addition to generating timeseries through the two-step procedure,
OASP may be used for generating the complex CW field over a
rectangular grid in range and depth. Here it is important to note that
when OASP is used with option 'O' or with automatic
sampling enabled, the transfer functions are computed for complex
frequencies. Complex frequency corresponds to applying a {\em
time-domain damping} which cannot be directly compensated for in the
transfer functions. However, real frequencies can be forced in
automatic sampling mode by using option 'J' (Version 2.1 and later).

Also, in version 2.1 and later, the postprocessor PP has been expanded
with a transmission loss option which converts the transfer function
to transmission losses plotted in the standard OAST forms of TL vs
range or depth-range contours. Here it is obviously important to use option 'J'
together with the automatic sampling. Otherwize the losses will be
overestimated. Also note that the automatic sampling works differently
from OAST's. Thus, OASP will use the selected time window to select a
wavenumber sampling which eliminates time-domain wrap-around. This
feature may actually be used for convergence tests, by systematically
increasing the time window $ (NX \times DT) $ to allow reduced
wavenumber sampling.
  
\subsection{Input Files for OASP}

The input files for OASP is structured in 8 blocks, as outlined in
Tables\,\ref{tab:oaspI}. In the following we
describe the significance of the various blocks, with particular
emphasis on differences between SAFARI-FIPP and OASP. 

\begin{table}
\begin{center}
\small
\begin{tabular}{|l|l|c|c|}
\hline \hline
Input parameter & Description & Units & Limits \\
\hline \hline
\multicolumn{4}{|l|}{\bf BLOCK I: TITLE } \\
\hline
TITLE & Title of run  & - & $\leq$ 80 ch. \\
\hline
\multicolumn{4}{|l|}{\bf BLOCK II: OPTIONS} \\
\hline
A B C $\cdots$ & Output options & - & $\leq$ 40 ch. \\
\hline
\multicolumn{4}{|l|}{\bf BLOCK III: SOURCE FREQUENCY} \\
\hline
FRC,COFF,IT,VS,VR & FRC: Center frequency of source & Hz & $>0$ \\
 	& COFF: Integration contour offset & dB/$\Lambda$ & COFF$\geq 0$ \\
        & IT: Source pulse type (only for option d) & & \\
	& VS,VR: Sou./Rec. velocity  (only for option d) & m/s & \\
\hline
\multicolumn{4}{|l|}{\bf BLOCK IV: ENVIRONMENT} \\
\hline
NL 	& Number of layers, incl. halfspaces	& - & NL$\geq 2$  \\
D,CC,CS,AC,AS,RO,RG,CL & D: Depth of interface. & m & - \\
.	& CC: Compressional speed & m/s & CC$\geq 0$ \\
.	& CS: Shear speed & m/s & - \\
.	& AC: Compressional attenuation & dB/$\Lambda$ & AC$\geq 0$ \\
.	& AS: Shear attenuation & dB/$\Lambda$ & AS$\geq 0$ \\
	& RO: Density 	& g/cm$^{3}$ & RO$\geq 0$ \\
	& RG: RMS value of interface roughness & m & - \\
	& CL: Correlation length of roughness & m & CL$>0$ \\
	& M:  Spectral exponent &   & > 1.5 \\
\hline
\multicolumn{4}{|l|}{\bf BLOCK V: SOURCES} \\
\hline
SD,NS,DS,AN,IA,FD,DA & SD: Source depth (mean for array) & m & - \\
	& NS: Number of sources in array & - &  NS$>0$ \\
	& DS: Vertical source spacing	& m & DS$>0$ \\
	& AN: Grazing angle of beam & deg & - \\
	& IA: Array type & - & $1 \leq $IA$\leq 5$ \\
	& FD: Focal depth of beam & m & FD$\neq$SD \\
        & DA: Dip angle. (Source type 4). & deg & - \\
\hline
\multicolumn{4}{|l|}{\bf BLOCK VI: RECEIVER DEPTHS} \\
\hline
RD1,RD2,NRD & RD1: Depth of first receiver & m & - \\
	& RD2: Depth of last receiver  & m & RD2$>$RD1 \\
	& NRD: Number of receiver depths & - & NR$>0$ \\
\hline
\multicolumn{4}{|l|}{\bf BLOCK VII: WAVENUMBER SAMPLING} \\
\hline
CMIN,CMAX & CMIN: Minimum phase velocity & m/s & CMIN$>0$ \\
	& CMAX: Maximum phase velocity & m/s & - \\
NW,IC1,IC2,IF & NW: Number of wavenumber samples & - & $>0$, -1 (auto) \\
	& IC1: First sampling point & - & IC1$\geq 1$ \\
	& IC2: Last sampling point & - & IC2$\leq$NW \\
        & IF: Freq. sample increment for kernels & & $\geq 0$ \\
\hline
\multicolumn{4}{|l|}{\bf BLOCK VIII: FREQUENCY AND RANGE  SAMPLING} \\
\hline
NT,FR1,FR2,DT,R1,DR,NR & NT: Number of time samples & & NT$=2^{M}$ \\
	& FR1: lower limit of frequency band & Hz & $\geq 0$ \\
	& FR2: upper limit of frequency band & Hz & $\geq$FR1 \\
        & DT: Time sampling increment & s & $> 0$ \\ 
        & R1: First range & km & \\
        & DR: Range increment & km & \\
	& NR: Number of ranges & & $> 0$ \\
\hline
\end{tabular}
\end{center}
\caption{Layout of OASP input files: Computational parameters.
	\label{tab:oaspI} }
\end{table} 

\subsection{Block I: Title}

The title printed on all graphic output generated by OASP.

\subsection{Block II: OASP options}

    OASP is downward compatible with SAFARI  Version 
3.0  and higher, and therefore supports all options and  features 
described in the SAFARI manual, except for those affecting the
generation of time-series, i.e. the following SAFARI options are 
\underline{not}
\underline{supported} by OASP:
\begin{description}
    \item[D] Depth stacked seismograms
    \item[S] Range stacked seismograms
    \item[R] Plot of source pulse
    \item[n] Source pulse type
\end{description}

    In  addition  to improved speed and  stability,  OASP  offers 
several new options. The currently supported options are:
\begin{itemize}
    \item[{\bf C}] Creates an $ \omega - k$ representation of the
      field in the form of contours of integration kernels as function
      of horizontal wavenumber (slowness if option {\bf B} is
      selected) and frequency (logarithmic y-axis). All axis
      parameters are determined automatically.  
\item[{\bf H}]
      Horizontal (radial) particle velocity calculated.  
\item[{\bf J}] Complex wavenumber contour. The contour is shifted into the
      upper halfpane by an offset controlled by the input parameter
      COFF (Block III). NOTE: If this option is used together with
      automatic sampling, the complex frequency integration (option
      {\bf O}) is disabled, allowing for computation of complex CW
      fields or transmission losses (plotted using PP). 
\item[{\bf K}]
      Computes the bulk pressure. In elastic media the bulk pressure only
      has contributions from the compressional potential. In fluid
      media the bulk pressure is equal to the acoustic pressure.
      Therefore for fluids this option yields the negative of the result
      produced by option N or R.  
\item[{\bf L}] Linear vertical source array.
\item[{\bf N}] Normal stress $\sigma_{zz}$ ($=-p$ in fluids)
      calculated.  
\item[{\bf O}] Complex frequency integration
      contour. This new option is the frequency equivalent of the
      complex wavenumber integration ({\bf J} option in OAST). It
      moves the frequency contour away from the real axis by an amount
      reducing the time domain wrap-around by a factor 50
      \cite{jkps}. This option can yield significant computational
      savings in cases where the received signal has a long time
      duration, and only the initial part is of interest, since it
      allows for selection of a time window shorter than the actual
      signal duration. Note that only wrap around from later times is
      reduced; therefore the time window should always be selected to
      contain the beginning of the signal!  
\item[{\bf P}] Plane
      geometry. The sources will be line-sources instead of
      point-sources as used in the default cylindrical geometry.
\item[{\bf R}] Computes the radial normal stress $\sigma_{rr}$
      (or $\sigma_{xx}$ for plane geometry).  \item[{\bf S}] Computes
      the stress equivalent of the shear potential in elastic
      media. This is an angle-independent measure, proportional to the
      shear potential, with no contribution from the compressional
      potential (incontrast to shear stress on a particular plane).
      For fluids this option yields zero.  
\item[{\bf T}] The new
      option `T' allows for specification of an array tilt in the
      vertical plane containing the source and the receivers.  See
      below for specification of array tilt parameters.  
\item[{\bf U}] Decomposed seismograms.  This option generates 5 transfer
      function files to be processed by PP:

          \begin{tabular}{ll}
          File name & Contents \\ \hline
          input.trf &     Complete transfer functions \\
          input.trfdc &    Downgoing compressional waves \\
          input.trfuc  &  Upgoing compressional waves  \\
          input.trfds  &  Downgoing  shear waves alone \\
          input.trfus &    Upgoing shear waves
          \end{tabular}
\item[{\bf V}] Vertical particle velocity calculated.
\item[{\bf Z}] Plot of SVP will be generated.

\item[{\bf d}] Radial {\em Doppler shift} is accounted for by
specifying  this option, using the theory developed by Schmidt and Kuperman
\cite{sk:jasa94}.  The source pulse and the radial projections of the
source and  receiver velocities must be specified in the input file
following the specification of the centre frequency and the contour
offset (Block II). Since this option requires incorporation of the
source function in the wavenumber integral, the PP post-processor
\underline{must} be used with source pulse -1 (impulse response).
    \item[{\bf f}] A full Hankel transform integration scheme is used
for low values of $ kr $ and tapered into the FFP integration used for
large $ kr $. The compensation is achieved at insignificant
additional computational cost and is recommended highly for cases
where the near field is  needed.    
          \item[{\bf g}] Rough interfaces 
          are assumed to be characterized by a Goff-Jordan power
          spectrum rather than the default Gaussian (Same as G).
          \item[{\bf l}] User defined source array. This new option is
          similar to option {\bf L} in the sense that that it
          introduces a vertical source array of time delayed sources
          of identical type. However, this option allows the depth,
          amplitude and delay time to be be specified individually for
          each source in the array. The source data should be provided
          in a separate file, {\bf input.src}, in the format described
          below in Section~\ref{oaspsar}.  
\item[{\bf r}] Computes {\em Reflectivity Seismograms}, i.e. no source contyributions included.
\item[{\bf s}] Outputs the
          mean field discontinuity at a rough interface to the file
          'input'.rhs for input to the time domain reverberation model
          OASSP.  
\item[{\bf t}] Eliminates the wavenumber integration
          and computes transfer functions for individual slowness
          components (or plane wave components). The Fourier transform
          performed in PP will then directly compute the
          slowness/intercept-time or $\tau - p$ response for each of
          the selected depths. When option {\bf t} is selected, the
          range parameters in the data file are insignificant.
\item[{\bf v}] As option {\bf l} this option allows for
          specifying a non-standard source array. However, it is more
          general in the sense that different types of sources can be
          applied in the same array, and the sources can have
          different signatures. The array geometry and the complex
          amplitudes are specified in a file {\bf input.strf} which
          should be of {\bf trf} format as described in
          Section~\ref{oaspsar}. 
 \item[{\bf \#}] Number $(1-5)$
          specifying the source type (explosive, forces, seismic
          moment) as described in Section~\ref{oaspsou}
\end{itemize}


\subsubsection{Block III: Source Frequency}

FRC is the source center frequency. As the source convolution is
performed in PP,  FRC is not used in OASP, but will be written to the
transfer function file and become the default for PP.

COFF is the complex wavenumber
integration 
contour offset. To be specified in
		$dB/\lambda$, where $\lambda$ is the wavelength at the source 
		depth SD. As only the horizontal part of
		the integration contour is considered, this parameter
		should not be chosen so large, that the amplitudes
		at the ends of the integration interval become 
		significant. In lossless cases too small values
		will give sampling problems at the normal modes
		and other singularities. For intermediate values,
		the result is independent of the choice of COFF,
		but a good value to choose is one that gives 60 dB
		attenuation at the longest range considered in the FFT, i.e.
                \begin{displaymath}
		\mbox{COFF} = \frac{60 \ast \mbox{CC(SD)}}{( \mbox{FREQ} \ast \mbox{R}_{max} )}
		\end{displaymath} 
       		where the maximum FFT range is
       		\begin{displaymath}
		\mbox{R}_{max} = \frac{\mbox{NP}}{\mbox{FREQ}\ast(1/\mbox{CMIN} - 1/\mbox{CMAX})}
		\end{displaymath}
		This value is the default which is applied if COFF
		is specified to 0.0.

\noindent{\bf Doppler shift}

By specifying option {\bf d} in OASP V.1.7 and higher,
radial {\em Doppler shift} is accounted for 
using the theory developed by Schmidt and Kuperman
\cite{sk:jasa94}.  The source pulse and the radial projections of the
source and  receiver velocities must be specified in the input file
following the specification of the centre frequency and the contour
offset (Block II), i.e.

\begin{tabular}{cc}
   Standard   &    For option {\bf d} \\ \hline
FRC COFF & IT VS VR
\end{tabular}


\tt IT \rm is a number identifying the source pulse as described
in Sec.\,\ref{postproc}. \tt VS \rm and \tt VR \rm are the
projected radial velocities in m/s of the source and receiver, respectively,
both being positive in the direction from source to receiver.
Since this option requires incorporation of the
source function in the wavenumber integral, the PP post-processor
\underline{must} subsequently be used with source pulse -1 (impulse response).


\subsubsection{Block IV: Environmental Model}

OASP supports all the environmental models allowed for SAFARI as well as the
ones described above in Section~\ref{oas_env}.  The significance of
the standard environmental parameters is as follows
\begin{itemize}	
		\item[NL:]	Number of layers, including the upper and lower
		half-spaces. These should always be included,
		even in cases where they are vacuum.

		\item[D:]	Depth in $m$ of upper boundary of layer or
		halfspace. The reference depth can be choosen
		arbitrarily, and D() is allowed to be negative.
		For layer no. 1, i.e. the upper half-space, this
     		parameter is dummy.

		\item[CC:]	Velocity of compressional waves in $m/s$.
        	If specified to 0.0, the layer or half-space is
		vacuum.

		\item[CS:]	Velocity of shear waves in m/s.
 		If specified to 0.0, the layer or half-space is fluid.
                If CS()$< 0$, it is the compressional velocity at bottom of
		layer, which is treated as fluid with $1/c(z)^{2}$ linear.

		\item[AC:]   Attenuation of compressional waves in 
		$dB/\lambda$. If the layer is fluid, and AC() is specified to
		0.0, then an imperical water attenuation is
		used (Skretting \& Leroy).

		\item[AS:]   Attenuation of shear waves in $dB/\lambda$

		\item[RO:]   Density in $g/cm^{3}$.

		\item[RG:]  RMS roughness of interface in $m$. RG(1) is dummy. If RG$<0$ it represents the negative of the RMS roughness, and the associated correlation length CL and the spectral exponent M should follow. If RG$>0$ the correlation length is assumed to be infinite.
		\item[CL:] Roughness correlation length in m. 

                \item[M:] Spectral exponent of the power spectrum as
                defined by Turgut \cite{Turgut_97}, with $1.5 <
                \mbox{M} \le 2.5 $ for realistic surfaces, with
                $\mbox{M} =1.5 $ corresponding to the highest
                roughness, and $\mbox{M}=2.5$ being a very smooth
                variation. For 2-D Goff-Jordan surfaces, the fractal
                dimension is $ \mbox{D} = 4.5 - \mbox{M} $
                Insignificant
                for Gaussian spectrum (option {\bf g} not specified)
                but a  value must
                be given.
		\end{itemize}


\subsubsection{Block V: Sources}

    OASP supports the same sources as  SAFARI-FIPP,  i.e 
explosive sources in fluids or solids or vertical point forces in 
solids  (option  X).  Multible sources in a  vertical  array  are 
supported. If sources with horizontal directionality are desired, 
the 3-dimensional version OASP3D must be used.

\noindent {\bf Source Types}
\label{oaspsou}  

As in SAFARI the default source type in OASP 
is an explosive type compressional source. In addition to the optional
vertical and horizontal point forces, various seismic moment sources have  been
added to OASP. The source type is specified by a number $(1-5)$ in the
option field (line 2). The translation is as follows:
\begin{enumerate}
\item Explosive source (default) normalized to unit pressure at 1 m distance.
\item Vertical point force with amplitude 1 N.
\item Horizontal (in-plane) point force with amplitude 1 N.
\item Dip-slip source with seismic moment 1 Nm. Dip angle specified in 
      degrees in block V, following the other parameters.
\item Omnidirectional seismic moment source representing explosive source. Same as type 1, but all three force dipoles have seismic moment 1 Nm.
\end{enumerate}

\noindent {\bf Source Normalization}

    In  SAFARI-FIPP,  the source pulse shape was defined  as  the 
pressure pulse produced at a distance of 1 m from the source (for 
solids the negative of the normal stress 1 m below the source). 

    In  OASP, the same source normalization has been  maintained 
for point sources (explosive sources) in fluid media.
    For solid media, however, the sources are normalized to  unit 
volume (1 m$^3$) injection for explosive sources and unit force 
1 N for point sources or 1 N/m for line sources.

\noindent {\bf User defined Source Arrays}
\label{oaspsar}

Version 1.6 of OASP has been upgraded to allow a user-defined source
array through options {\bf l} and {\bf v}. 

Option {\bf l} is intended
for general physical arrays with uneven spacing or special shadings,
As for the built-in arrays, such user-defined arrays may be present in
fluid as well as elastic media. The source type is specified as
described above, and the array geometry and shading should be given
in the file {\bf input.src} in the following format

\small
\begin{verbatim}
LS
SDC(1)  SDELAY(1)  SSTREN(1)    # Depth (m), Delay (s), Amplitude
SDC(2)  SDELAY(2)  SSTREN(2) 
SDC(3)  SDELAY(3)  SSTREN(3) 
  :        :          :
  :        :          :
SDC(LS) SDELAY(LS) SSTREN(LS) 
\end{verbatim}
\normalsize


Option {\bf v} is more general in the
sense that it allows for different source types to be mixed in the
array, and the pure time delay is replaced by a specification of the
complex amplitudes in the frequency domain, allowing for
representation of multibles etc. This option is used for {\em virtual
arrays} such as those imposed by coupling of wave systems. For
example, 
this option
is used for coupling tube wave phenomenae to propagation in a
stratified formation when modeling borehole seismics. Option {\bf v}
is only allowed for source arrays in elastic media (including
transversily isotropic layers). The complex amplitudes of the source
array is specified in the file {\bf input.strf}. This should be an
ASCII {\bf trf} file, and the frequency sampling should be consistent
with the frequency sampling selected in the input file {\bf input.dat}.
The sources are omnidirectional in the
horizontal.
The source type is identified by a type number in the file header, 
and each depth can
have one of each source type present. The possible source types are:
\begin{itemize}
\item[{\bf 10}] Seismic monopole, i.e. 3 perpendicular and identical
force dipoles. The unit is seismic moment (Nm).
\item[{\bf 11}] Vertical force dipole. The unit is seismic moment
(Nm).
\item[{\bf 12}] Vertical force, positive downwards. The unit is force
(N). This source type is e.g. used to represent caliber changes in fluid filled boreholes, as discussed by Kurkjian etal. \cite{Kurkjian_94}. 
\item[{\bf 13}] Tube wave source. The strf file contains the tube wave
pressure dependent term in Eq. (5) and (6) in the paper by Kurkjian
etal. \cite{Kurkjian_94},
\begin{displaymath}
\frac{2 \pi a^2 \alpha_T}{-i \omega} p(\omega),
\end{displaymath}
where $p(\omega)$ is the complex pressure at angular frequency $\omega$, $a$ is the tube radius, and $\alpha_T$ is the tube wave speed at the source depths. 
\end{itemize}
The source types are recognized by PP which can therefore be used to
check your source timeseries by simply specifying {\bf input.strf} in
Field~1 of the PP main menu. 
An example of an {\bf strf}-file for 21 source depths, with a monopole
and a dipole source  at each depth, is

\small
\begin{verbatim}
 PULSETRF                               # TRF file identification
 OASP16                                 # Calculating program
           2                            # No. sources per depth NSIN
          10          11                # Source types
 tube wave simulation                   # Title
 +                                      # Sign if time factor exponent
 400.0                                  # Center frequency 
   6.0                                  # Depth of primary source
  -0.5  19.5  21                        # SD-min, SD-max, LS
   0.0   0.0   1                        # Range (fixed).
  1024   2   104  0.0001                # Time/frequency parameters
           1                            # Dummy
  -38.20335                             # Imag. part of frequencies
           1                            # One Fourier order (fixed)
           1                            # Fixed
           0                            # Dummy
           0                            # Dummy
           0                            # Dummy
  0.0                                   # Dummy
  0.0                                   # Dummy
  0.0                                   # Dummy
  0.0                                   # Dummy
  0.0                                   # Dummy
  -24.15950 82.01517 -24.15950 82.01517 # Data
  -28.20251 83.04697 -28.20251 83.04697 # Data
  -32.38960 83.93490 -32.38960 83.93490 # Data
  -36.71837 84.66933 -36.71837 84.66933 # Data
  -41.18590 85.24049 -41.18590 85.24049 # Data
       :        :         :        :        :
\end{verbatim}
\normalsize

\noindent {\bf Note:} All lines should start with an empty space!.
 The time/frequency parameters are given in the form

\small
\begin{verbatim}
 NT  LX  MX  DT
\end{verbatim}
\normalsize
\noindent where
\begin{itemize}
\item[NT] is the number of time samples
\item[DT] is the time sampling interval in seconds
\item[LX] is the index of the first frequency, LX = INT(FR1*DT)
\item[MX] is the index of the last frequency, MX = INT(FR2*DT)
\end{itemize}

\noindent The complex data must be written in the following loop structure

\small
\begin{verbatim}
      DO 10 K = LX,MX
       DO 10 L = 1,LS
        WRITE(15,*) (REAL(TRF(K,L,M)),AIMAG(TRF(K,L,M)), M=1,NSIN)
 10   CONTINUE
\end{verbatim}
\normalsize

\subsubsection{Block VI: Receivers}

The default specification of the receiver depths is the same as for
SAFARI, i.e. through the parameters RD1, RD2 and NRD in Block VI, with
\begin{itemize}
\item[RD1]  Depth of uppermost receiver in meters
\item[RD2]  Depth of lowermost receiver in meters
\item[NRD]   Number of receiver depths 
\end{itemize}

The NRD receivers are placed equidistantly in the vertical.

\noindent {\bf Non-equidistant Receiver Depths}

In OASES the receiver depths can optionally be specified individually.
The parameter NRD is used as a flag for this option. Thus, if NRD $< 0$
the number of receivers is interpreted as --NRD, with the individual
depths following immidiately following Block VI. As an example, SAFARI
FIPP case 2 with receivers at 100, 105 and 120 m is run with the
following data file: 

\small
\begin{verbatim}
SAFARI-FIPP case 2.
V H F O			       # Complex frequency integration. 
5 0
4
  0    0     0 0   0   0   0
  0 1500     0 0   0   1   0
100 1600   400 0.2 0.5 1.8 0
120 1800   600 0.1 0.2 2.0 0
95
100 100 -3                     # 3 receivers
100.0 105.0 120.0              # Receiver depths in meters
300 1E8                      
1024 1 950 0
2048 0.0 12.5 0.006 0.5 0.5 5
\end{verbatim}
\normalsize 

The PP Post-processor is compatible and will depth-stack the traces at
the correct depths.

\noindent {\bf Tilted Receiver Arrays}

The new option `T' allows for specification of an array tilt in the
vertical plane containing the source and the receivers.

The tilt angle and rotation origin is specified in the receiver depth
line (Block VI):

\begin{tabular}{cc}
   Standard   &    For option T \\ \hline
RD1 RD2 NR &  ZREF  ANGLE
\end{tabular}


The vertical arrays are rotated by an angle `ANGLE' in deg relative
to the vertical. The rotation is performed with origin at depth `ZREF'.

The parameters RD1, RD2 and NR always refer to the untilted case. 
In the tilted case these parameters do therefore define the array
geometry and not the actual depths of the receivers in the tilted array.
The same is the case for the graphics output produced by the
post-processor PP. 


The source(s) is always at the origin and is therefore not rotated.
Thus, for zero-offset tilted VSP-s, the reference depth ZREF should be set 
equal to the source depth SD!



\subsubsection{Block VII: Wavenumber integration}

This block specifies the wavenumber sampling in the standard SAFARI
format, with the significance of the parameters being as follows:
\begin{itemize}
		\item[CMIN:]   Minimum phase velocity in m/s. Determines the
		upper limit of the truncated horizontal wavenumber space:
			\begin{displaymath}
			k_{max} = \frac{2\pi \ast \mbox{FREQ}}{\mbox{CMIN}}
			\end{displaymath}
		\item[CMAX:]	Maximum phase velocity in m/s. Determines the
		lower limit of the truncated horizontal wave-
		number space:
			\begin{displaymath}
			k_{min} = \frac{2\pi \ast \mbox{FREQ}}{\mbox{CMAX}}
			\end{displaymath}
		In plane geometry ( option P ) CMAX may be specified as 
		negative. In this case, the negative wavenumber spectrum
		will be included with $k_{min}=-k_{max}$, yielding correct 
		solution also at zero range. In contrast to SAFARI,
		OASP allows for complex contour integration (option J)
		in this case.
		\item[NW:]	Number of sampling points in wavenumber space.
		In contrast to what is the case for OAST, 
		NW does here not have to be an integer  power of 2.				The sampling points are placed equidistantly
		in the truncated wavenumber space determined
		by CMIN and CMAX. If CMAX$<0$, i.e. the inclusion of
		the negative spectrum is enabled, then the NW sample
		points will be distributed along the positive
		wavenumber axis only, with the negative
		components obtained by symmetry.

		\item[IC1:]  Number of the first sampling point where the
		calculation is to be performed. If IC1$>1$, 
		then the Hankel transform is Hanning-windowed in the
		interval 1,2$\ldots$IC1-1 before integration.

		\item[IC2:]	Number of the last sampling point where the 
		calculation is to be performed. If IC2$<$NWN,
		then the Hankel transform is Hanning windowed in the
		interval IC2+1,$\ldots$NW before integration.

		\item[IF:] Frequency increment for plotting of
integration  kernels. A value of 0 disables the plotting.
		\end{itemize}
 
\noindent {\bf Automatic wavenumber sampling}



    OASP Version  1.4 and higher  have been supplied with  an  
automatic  sampling 
feature,  making  it possible for inexperienced users  to  obtain 
correct   answers  in  the  first  attempt  without   the   usual 
convergence  testing.  The  automatic sampling  is  activated  by 
specifying the parameter NW to $ -1$ and it automatically  activates 
the  complex frequency integration contour even though option  {\bf O} 
may  not have been specified. The parameters IC1 and IC2 have  no 
effect if the automatic sampling is selected.
 
As an example, to run the SAFARI-FIPP case 2 problem with automatic
sampling, change the data file as  follows:

\small
\begin{verbatim}
SAFARI-FIPP case 2. Auto sampling.
V H f                         # Bessel integration.
5 0
4
  0    0     0 0   0   0   0
  0 1500     0 0   0   1   0
100 1600   400 0.2 0.5 1.8 0
120 1800   600 0.1 0.2 2.0 0
95
100 100 1 
 300 1E8                      
-1 0 0                         # NW   =   -1
2048 0.0 12.5 0.006 0.5 0.5 5
\end{verbatim}
\normalsize 

\subsection{Execution of OASP}

    As  for  SAFARI,  filenames  are  passed  to  the  code   via 
environmental parameters. In Unix systems a typical command  file 
{\bf oasp} (in  \$HOME/oases/bin) is:

\small
\begin{verbatim}
    #                            the number sign invokes the C-shell 
    setenv FOR001 $1.dat       # input file 
    setenv FOR002 $1.src       # Source array input file
    setenv FOR015 $1.strf      # Source array trf file
    setenv FOR019 $1.plp       # plot parameter file
    setenv FOR020 $1.plt       # plot data file  
    setenv FOR028 $1.cdr       # contour plot parameter file 
    setenv FOR029 $1.bdr       # contour plot data file 
    setenv FOR045 $1.rhs       # mean field amplitudes at rough interface
    setenv FOR046 $1.vol       # mean field in volume scattering layer
    oasp2                      # executable
\end{verbatim}
\normalsize

    After preparing a data file with the name {\bf input.dat}, OASP  is 
executed by the command:

    $>$ {\bf oasp input}


\subsection{Graphics}  

    Command files are provided in a path directory for generating 
the graphics.

\noindent    To generate curve plots, issue the command:

    $>$ {\bf mplot input}

\noindent    To generate contour plots, issue the command:

    $>$ {\bf cplot input}

\newpage
\subsection{OASP - Examples}

\subsubsection{OASP Transmission loss calculation}

\figmithtwo{figs/free.ps}{6.5}{figs/free_f.ps}{6.5}
{Transmission loss contours produced by OASP and PP for a point source
in an infinite fluid medium. a) Default trapezoidal rule integration
with fast-field Hankel function approximation. b) Same problem, but
full bessel function integration (Option f)}
{fig:free}

As an example of the use of OASP for computing transmission losses,
and for demonstrating the difference between the full Bessel function
integration and the default fast-field Hankel function approximation,
the following data files computes the transfer function for a 150 Hz
simple point source in a lossless, infinite fluid medium.

\begin{verbatim}
free.dat                           |   free_f.dat
_____________________________________________________________________
Infinite medium.                   |   Inf. medium. OASP Bes.Int.
N J                                |   N J f  
150 0                              |   150 0
2                                  |   2
0 1500 0 0.1 0 1 0                 |   0 1500 0 0.1 0 1 0
0 1500 0 0.1 0 1 0                 |   0 1500 0 0.1 0 1 0
0                                  |   0
-200 200 61 30                     |   -200 200 61 30
1000 1E8                           |   1000 1E8
-1 1 1 1                           |   -1 1 1 1
1024 150 150 0.0005 0 0.005 81     |   1024 150 150 0.0005 0 0.005 81  
\end{verbatim}

The execution of OASP with these input files generates the transfer
function files, which are then converted to transmission loss by
PP. Figure\,\ref{fig:free} shows the transmission loss contours in
depth and range, plotted from PP using \tt plotmtv. The near field
differences between the 'exact' and the fast-field Hankel transforms
are evident.




